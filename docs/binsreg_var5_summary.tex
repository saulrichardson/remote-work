\documentclass{article}
\usepackage[margin=1in]{geometry}
\usepackage{amsmath}
\usepackage{graphicx}
\begin{document}

\section*{Binsreg Visualization of the Startup Remote Effect}

\paragraph{Command.}
To stay aligned with the canonical specification, we:
\begin{enumerate}
    \item Restricted the panel to startup observations (\texttt{startup == 1}).
    \item Ran the fixed-effects regression
\begin{verbatim}
reghdfe total_contributions_q100 var3 var4 var6 var7, \\
        absorb(user_id firm_id yh) vce(cluster firm_id) resid
\end{verbatim}
    which partials out user, firm, and period fixed effects while controlling for \texttt{var3}, \texttt{var4}, \texttt{var6}, and \texttt{var7}. (where `var3 = remote × covid`, `var4 = covid × startup`, `var6 = covid × company\_teleworkable`, and `var7 = startup × covid × company\_teleworkable`) The \texttt{resid} option provides the Frisch--Waugh residuals (\texttt{contrib\_resid}) that match the table estimates.
    \item Fed those residuals into the binscatter:
\begin{verbatim}
binsreg contrib_resid age, by(treat_var5) samebinsby \\
        nbins(10) binspos(es) masspoints(off) \\
        dots(1 0) ci(1 0) line(1 0)
\end{verbatim}
    where \texttt{treat\_var5 = (var5 > 0)} flags remote $\times$ COVID exposure for startups. The command exports both a Stata figure (\texttt{results/final/figures/binsreg\_var5\_levels\_simple.png}) and the underlying dataset (\texttt{results/final/binsreg\_var5\_levels.dta}) for post-processing.
\end{enumerate}

\paragraph{Meaning.}
The two binsreg curves plot FE-adjusted contribution ranks for \emph{startup} observations only:
\begin{itemize}
    \item ``Baseline'' (blue) are startup periods with $\texttt{var5}=0$---pre-COVID or in-person spells.
    \item ``Remote $\times$ COVID'' (red) are the same startups once $\texttt{var5}>0$, i.e., remote exposure in the post-COVID window.
\end{itemize}
Because both curves are computed after partialing out the fixed effects and controls, the vertical distance between them at any age is the nonparametric analogue of the regression's $\beta_{\texttt{var5}}$. A flat gap would indicate a constant treatment effect; allowing bins lets us see heterogeneity across firm age.


\begin{figure}[ht]
    \centering
    \includegraphics[width=0.85\textwidth]{../results/final/figures/effect_var5_age_python.png}
    \caption{Startup remote effect (binsreg residuals)}
    \label{fig:binsreg_var5}
\end{figure}

\paragraph{Interpretation.}
Figure \ref{fig:binsreg_var5} shows:
\begin{itemize}
    \item For the youngest startups (age $\approx 1$), untreated residuals drop sharply ($\approx-12$), whereas treated residuals sit slightly above zero.
    \item As startups age (2--5 years), both curves converge toward zero; the treatment advantage shrinks to a few tenths of a rank, and by ages 6--10 the lines overlap within their confidence intervals.
\end{itemize}

\end{document}
