\documentclass[11pt]{article}
\usepackage{booktabs}
\usepackage{array}
\usepackage{multirow}
\usepackage{graphicx}
\usepackage[margin=1in]{geometry}
\usepackage{adjustbox}
\usepackage{threeparttable}
\usepackage{amssymb}
\usepackage{amsmath}

\title{Growth Mechanisms and Fixed Effects Specifications\\
\large Remote Work Productivity Effects}
\author{Analysis Report}
\date{\today}

\begin{document}

\maketitle

\section{Introduction}

This document presents a comprehensive analysis of how firm growth interacts with remote work effects on productivity. We examine two key dimensions:
\begin{enumerate}
\item \textbf{Growth Types}: Endogenous (raw) vs. Exogenous (predicted by external factors)
\item \textbf{Fixed Effects}: Worker-Firm pairs vs. Separate worker and firm effects
\end{enumerate}

\section{Methodology}

\subsection{Empirical Specification}

We estimate:
\begin{equation}
Y_{it} = \beta_1 \text{Remote}_i \times \text{Post}_t + \beta_2 \text{Remote}_i \times \text{Post}_t \times \text{Startup}_i + \gamma X_{it} + \alpha + \epsilon_{it}
\end{equation}

where $\alpha$ represents either:
\begin{itemize}
\item Worker-firm fixed effects ($\alpha_{i,f}$) - controlling for time-invariant match quality
\item Separate worker ($\alpha_i$) and firm ($\alpha_f$) fixed effects
\end{itemize}

\subsection{Growth Measures}

\begin{enumerate}
\item \textbf{Endogenous Growth}: Raw post-COVID firm growth (above/below median)
\item \textbf{Exogenous Growth}: Growth predicted by industry trends, MSA trends, rent, and HHI
\end{enumerate}

The exogenous growth captures firm performance driven by external factors beyond the firm's control, isolating the growth component orthogonal to firm-specific advantages.

\section{Results}

\subsection{Main Specification: Worker-Firm Fixed Effects}

% Include the original table
\input{../results/cleaned/growth_mechanisms_focused.tex}

\subsection{Comparing Fixed Effects Specifications}

% Include the comparison table
\documentclass[11pt]{article}
\usepackage{booktabs}
\usepackage{array}
\usepackage{multirow}
\usepackage{graphicx}
\usepackage[margin=1in]{geometry}
\usepackage{adjustbox}
\usepackage{threeparttable}
\usepackage{amssymb}
\usepackage{amsmath}

\title{Growth Mechanisms and Fixed Effects Specifications\\
\large Remote Work Productivity Effects}
\author{Analysis Report}
\date{\today}

\begin{document}

\maketitle

\section{Introduction}

This document presents a comprehensive analysis of how firm growth interacts with remote work effects on productivity. We examine two key dimensions:
\begin{enumerate}
\item \textbf{Growth Types}: Endogenous (raw) vs. Exogenous (predicted by external factors)
\item \textbf{Fixed Effects}: Worker-Firm pairs vs. Separate worker and firm effects
\end{enumerate}

\section{Methodology}

\subsection{Empirical Specification}

We estimate:
\begin{equation}
Y_{it} = \beta_1 \text{Remote}_i \times \text{Post}_t + \beta_2 \text{Remote}_i \times \text{Post}_t \times \text{Startup}_i + \gamma X_{it} + \alpha + \epsilon_{it}
\end{equation}

where $\alpha$ represents either:
\begin{itemize}
\item Worker-firm fixed effects ($\alpha_{i,f}$) - controlling for time-invariant match quality
\item Separate worker ($\alpha_i$) and firm ($\alpha_f$) fixed effects
\end{itemize}

\subsection{Growth Measures}

\begin{enumerate}
\item \textbf{Endogenous Growth}: Raw post-COVID firm growth (above/below median)
\item \textbf{Exogenous Growth}: Growth predicted by industry trends, MSA trends, rent, and HHI
\end{enumerate}

The exogenous growth captures firm performance driven by external factors beyond the firm's control, isolating the growth component orthogonal to firm-specific advantages.

\section{Results}

\subsection{Main Specification: Worker-Firm Fixed Effects}

% Include the original table
\input{../results/cleaned/growth_mechanisms_focused.tex}

\subsection{Comparing Fixed Effects Specifications}

% Include the comparison table
\documentclass[11pt]{article}
\usepackage{booktabs}
\usepackage{array}
\usepackage{multirow}
\usepackage{graphicx}
\usepackage[margin=1in]{geometry}
\usepackage{adjustbox}
\usepackage{threeparttable}
\usepackage{amssymb}
\usepackage{amsmath}

\title{Growth Mechanisms and Fixed Effects Specifications\\
\large Remote Work Productivity Effects}
\author{Analysis Report}
\date{\today}

\begin{document}

\maketitle

\section{Introduction}

This document presents a comprehensive analysis of how firm growth interacts with remote work effects on productivity. We examine two key dimensions:
\begin{enumerate}
\item \textbf{Growth Types}: Endogenous (raw) vs. Exogenous (predicted by external factors)
\item \textbf{Fixed Effects}: Worker-Firm pairs vs. Separate worker and firm effects
\end{enumerate}

\section{Methodology}

\subsection{Empirical Specification}

We estimate:
\begin{equation}
Y_{it} = \beta_1 \text{Remote}_i \times \text{Post}_t + \beta_2 \text{Remote}_i \times \text{Post}_t \times \text{Startup}_i + \gamma X_{it} + \alpha + \epsilon_{it}
\end{equation}

where $\alpha$ represents either:
\begin{itemize}
\item Worker-firm fixed effects ($\alpha_{i,f}$) - controlling for time-invariant match quality
\item Separate worker ($\alpha_i$) and firm ($\alpha_f$) fixed effects
\end{itemize}

\subsection{Growth Measures}

\begin{enumerate}
\item \textbf{Endogenous Growth}: Raw post-COVID firm growth (above/below median)
\item \textbf{Exogenous Growth}: Growth predicted by industry trends, MSA trends, rent, and HHI
\end{enumerate}

The exogenous growth captures firm performance driven by external factors beyond the firm's control, isolating the growth component orthogonal to firm-specific advantages.

\section{Results}

\subsection{Main Specification: Worker-Firm Fixed Effects}

% Include the original table
\input{../results/cleaned/growth_mechanisms_focused.tex}

\subsection{Comparing Fixed Effects Specifications}

% Include the comparison table
\documentclass[11pt]{article}
\usepackage{booktabs}
\usepackage{array}
\usepackage{multirow}
\usepackage{graphicx}
\usepackage[margin=1in]{geometry}
\usepackage{adjustbox}
\usepackage{threeparttable}
\usepackage{amssymb}
\usepackage{amsmath}

\title{Growth Mechanisms and Fixed Effects Specifications\\
\large Remote Work Productivity Effects}
\author{Analysis Report}
\date{\today}

\begin{document}

\maketitle

\section{Introduction}

This document presents a comprehensive analysis of how firm growth interacts with remote work effects on productivity. We examine two key dimensions:
\begin{enumerate}
\item \textbf{Growth Types}: Endogenous (raw) vs. Exogenous (predicted by external factors)
\item \textbf{Fixed Effects}: Worker-Firm pairs vs. Separate worker and firm effects
\end{enumerate}

\section{Methodology}

\subsection{Empirical Specification}

We estimate:
\begin{equation}
Y_{it} = \beta_1 \text{Remote}_i \times \text{Post}_t + \beta_2 \text{Remote}_i \times \text{Post}_t \times \text{Startup}_i + \gamma X_{it} + \alpha + \epsilon_{it}
\end{equation}

where $\alpha$ represents either:
\begin{itemize}
\item Worker-firm fixed effects ($\alpha_{i,f}$) - controlling for time-invariant match quality
\item Separate worker ($\alpha_i$) and firm ($\alpha_f$) fixed effects
\end{itemize}

\subsection{Growth Measures}

\begin{enumerate}
\item \textbf{Endogenous Growth}: Raw post-COVID firm growth (above/below median)
\item \textbf{Exogenous Growth}: Growth predicted by industry trends, MSA trends, rent, and HHI
\end{enumerate}

The exogenous growth captures firm performance driven by external factors beyond the firm's control, isolating the growth component orthogonal to firm-specific advantages.

\section{Results}

\subsection{Main Specification: Worker-Firm Fixed Effects}

% Include the original table
\input{../results/cleaned/growth_mechanisms_focused.tex}

\subsection{Comparing Fixed Effects Specifications}

% Include the comparison table
\input{../results/cleaned/growth_mechanisms_both_fe.tex}

\section{Key Findings}

\subsection{Effect of Fixed Effects Specification}

\begin{enumerate}
\item \textbf{Baseline Effects}: 
   \begin{itemize}
   \item Worker-Firm FE: Startup advantage = 12.34 pp (p<0.05)
   \item Separate FE: Startup advantage = 9.86 pp (p<0.10)
   \end{itemize}
   The stronger effect with worker-firm FE suggests that controlling for match quality is important.

\item \textbf{Endogenous Growth}: 
   \begin{itemize}
   \item Worker-Firm FE: Effect drops to 6.78 pp (not significant)
   \item Separate FE: Effect drops to 4.74 pp (not significant)
   \end{itemize}
   Raw growth substantially weakens the startup advantage under both specifications.

\item \textbf{Exogenous Growth}: 
   \begin{itemize}
   \item Worker-Firm FE: Effect = 11.90 pp (p<0.10)
   \item Separate FE: Effect = 8.55 pp (not significant)
   \end{itemize}
   The startup advantage persists when controlling for externally-driven growth, especially with worker-firm FE.
\end{enumerate}

\subsection{Interpretation}

The results reveal several important insights:

\begin{enumerate}
\item \textbf{Match Quality Matters}: The worker-firm fixed effects specification consistently shows stronger startup advantages, suggesting that startups may be better at matching with remote-compatible workers.

\item \textbf{Growth Decomposition}: 
   \begin{itemize}
   \item Raw growth (endogenous) captures both firm-specific advantages and external factors
   \item When we control for raw growth, the startup advantage largely disappears
   \item However, when we control only for externally-driven growth (exogenous), the startup advantage persists
   \end{itemize}

\item \textbf{Mechanism}: The pattern suggests that startups' remote work advantage is correlated with their overall growth performance, but is not simply a mechanical result of being in growing industries or locations.
\end{enumerate}

\section{Robustness and Identification}

\begin{itemize}
\item All specifications use pre-determined remote work policies as instruments
\item First-stage F-statistics (>80) indicate strong instruments
\item Standard errors are clustered at the user level
\item The analysis focuses on the pre-COVID period to avoid confounding with pandemic-specific effects
\end{itemize}

\section{Conclusion}

The analysis demonstrates that:
\begin{enumerate}
\item Startups have a significant advantage in remote work productivity
\item This advantage is partially explained by their overall growth trajectory
\item However, the advantage persists when controlling for growth driven by external factors
\item Worker-firm match quality appears to be an important component of the startup advantage
\end{enumerate}

These findings suggest that startups' success with remote work reflects both selection (matching with remote-compatible workers) and treatment (better remote work practices) effects.

\end{document}

\section{Key Findings}

\subsection{Effect of Fixed Effects Specification}

\begin{enumerate}
\item \textbf{Baseline Effects}: 
   \begin{itemize}
   \item Worker-Firm FE: Startup advantage = 12.34 pp (p<0.05)
   \item Separate FE: Startup advantage = 9.86 pp (p<0.10)
   \end{itemize}
   The stronger effect with worker-firm FE suggests that controlling for match quality is important.

\item \textbf{Endogenous Growth}: 
   \begin{itemize}
   \item Worker-Firm FE: Effect drops to 6.78 pp (not significant)
   \item Separate FE: Effect drops to 4.74 pp (not significant)
   \end{itemize}
   Raw growth substantially weakens the startup advantage under both specifications.

\item \textbf{Exogenous Growth}: 
   \begin{itemize}
   \item Worker-Firm FE: Effect = 11.90 pp (p<0.10)
   \item Separate FE: Effect = 8.55 pp (not significant)
   \end{itemize}
   The startup advantage persists when controlling for externally-driven growth, especially with worker-firm FE.
\end{enumerate}

\subsection{Interpretation}

The results reveal several important insights:

\begin{enumerate}
\item \textbf{Match Quality Matters}: The worker-firm fixed effects specification consistently shows stronger startup advantages, suggesting that startups may be better at matching with remote-compatible workers.

\item \textbf{Growth Decomposition}: 
   \begin{itemize}
   \item Raw growth (endogenous) captures both firm-specific advantages and external factors
   \item When we control for raw growth, the startup advantage largely disappears
   \item However, when we control only for externally-driven growth (exogenous), the startup advantage persists
   \end{itemize}

\item \textbf{Mechanism}: The pattern suggests that startups' remote work advantage is correlated with their overall growth performance, but is not simply a mechanical result of being in growing industries or locations.
\end{enumerate}

\section{Robustness and Identification}

\begin{itemize}
\item All specifications use pre-determined remote work policies as instruments
\item First-stage F-statistics (>80) indicate strong instruments
\item Standard errors are clustered at the user level
\item The analysis focuses on the pre-COVID period to avoid confounding with pandemic-specific effects
\end{itemize}

\section{Conclusion}

The analysis demonstrates that:
\begin{enumerate}
\item Startups have a significant advantage in remote work productivity
\item This advantage is partially explained by their overall growth trajectory
\item However, the advantage persists when controlling for growth driven by external factors
\item Worker-firm match quality appears to be an important component of the startup advantage
\end{enumerate}

These findings suggest that startups' success with remote work reflects both selection (matching with remote-compatible workers) and treatment (better remote work practices) effects.

\end{document}

\section{Key Findings}

\subsection{Effect of Fixed Effects Specification}

\begin{enumerate}
\item \textbf{Baseline Effects}: 
   \begin{itemize}
   \item Worker-Firm FE: Startup advantage = 12.34 pp (p<0.05)
   \item Separate FE: Startup advantage = 9.86 pp (p<0.10)
   \end{itemize}
   The stronger effect with worker-firm FE suggests that controlling for match quality is important.

\item \textbf{Endogenous Growth}: 
   \begin{itemize}
   \item Worker-Firm FE: Effect drops to 6.78 pp (not significant)
   \item Separate FE: Effect drops to 4.74 pp (not significant)
   \end{itemize}
   Raw growth substantially weakens the startup advantage under both specifications.

\item \textbf{Exogenous Growth}: 
   \begin{itemize}
   \item Worker-Firm FE: Effect = 11.90 pp (p<0.10)
   \item Separate FE: Effect = 8.55 pp (not significant)
   \end{itemize}
   The startup advantage persists when controlling for externally-driven growth, especially with worker-firm FE.
\end{enumerate}

\subsection{Interpretation}

The results reveal several important insights:

\begin{enumerate}
\item \textbf{Match Quality Matters}: The worker-firm fixed effects specification consistently shows stronger startup advantages, suggesting that startups may be better at matching with remote-compatible workers.

\item \textbf{Growth Decomposition}: 
   \begin{itemize}
   \item Raw growth (endogenous) captures both firm-specific advantages and external factors
   \item When we control for raw growth, the startup advantage largely disappears
   \item However, when we control only for externally-driven growth (exogenous), the startup advantage persists
   \end{itemize}

\item \textbf{Mechanism}: The pattern suggests that startups' remote work advantage is correlated with their overall growth performance, but is not simply a mechanical result of being in growing industries or locations.
\end{enumerate}

\section{Robustness and Identification}

\begin{itemize}
\item All specifications use pre-determined remote work policies as instruments
\item First-stage F-statistics (>80) indicate strong instruments
\item Standard errors are clustered at the user level
\item The analysis focuses on the pre-COVID period to avoid confounding with pandemic-specific effects
\end{itemize}

\section{Conclusion}

The analysis demonstrates that:
\begin{enumerate}
\item Startups have a significant advantage in remote work productivity
\item This advantage is partially explained by their overall growth trajectory
\item However, the advantage persists when controlling for growth driven by external factors
\item Worker-firm match quality appears to be an important component of the startup advantage
\end{enumerate}

These findings suggest that startups' success with remote work reflects both selection (matching with remote-compatible workers) and treatment (better remote work practices) effects.

\end{document}

\section{Key Findings}

\subsection{Effect of Fixed Effects Specification}

\begin{enumerate}
\item \textbf{Baseline Effects}: 
   \begin{itemize}
   \item Worker-Firm FE: Startup advantage = 12.34 pp (p<0.05)
   \item Separate FE: Startup advantage = 9.86 pp (p<0.10)
   \end{itemize}
   The stronger effect with worker-firm FE suggests that controlling for match quality is important.

\item \textbf{Endogenous Growth}: 
   \begin{itemize}
   \item Worker-Firm FE: Effect drops to 6.78 pp (not significant)
   \item Separate FE: Effect drops to 4.74 pp (not significant)
   \end{itemize}
   Raw growth substantially weakens the startup advantage under both specifications.

\item \textbf{Exogenous Growth}: 
   \begin{itemize}
   \item Worker-Firm FE: Effect = 11.90 pp (p<0.10)
   \item Separate FE: Effect = 8.55 pp (not significant)
   \end{itemize}
   The startup advantage persists when controlling for externally-driven growth, especially with worker-firm FE.
\end{enumerate}

\subsection{Interpretation}

The results reveal several important insights:

\begin{enumerate}
\item \textbf{Match Quality Matters}: The worker-firm fixed effects specification consistently shows stronger startup advantages, suggesting that startups may be better at matching with remote-compatible workers.

\item \textbf{Growth Decomposition}: 
   \begin{itemize}
   \item Raw growth (endogenous) captures both firm-specific advantages and external factors
   \item When we control for raw growth, the startup advantage largely disappears
   \item However, when we control only for externally-driven growth (exogenous), the startup advantage persists
   \end{itemize}

\item \textbf{Mechanism}: The pattern suggests that startups' remote work advantage is correlated with their overall growth performance, but is not simply a mechanical result of being in growing industries or locations.
\end{enumerate}

\section{Robustness and Identification}

\begin{itemize}
\item All specifications use pre-determined remote work policies as instruments
\item First-stage F-statistics (>80) indicate strong instruments
\item Standard errors are clustered at the user level
\item The analysis focuses on the pre-COVID period to avoid confounding with pandemic-specific effects
\end{itemize}

\section{Conclusion}

The analysis demonstrates that:
\begin{enumerate}
\item Startups have a significant advantage in remote work productivity
\item This advantage is partially explained by their overall growth trajectory
\item However, the advantage persists when controlling for growth driven by external factors
\item Worker-firm match quality appears to be an important component of the startup advantage
\end{enumerate}

These findings suggest that startups' success with remote work reflects both selection (matching with remote-compatible workers) and treatment (better remote work practices) effects.

\end{document}