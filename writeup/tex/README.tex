% ======================================================================
% README.tex – Directory-focused repository documentation
% ----------------------------------------------------------------------
% This file is the authoritative description of how the repository is
% organised.  It compiles to README.pdf at the project root.
% ======================================================================

\documentclass[11pt]{article}

% ----------------------------------------------------------------------
% Packages
% ----------------------------------------------------------------------
\usepackage[margin=1in]{geometry}
\usepackage{hyperref}
\usepackage{xcolor}
\usepackage{enumitem}
\hypersetup{
  colorlinks=true,
  linkcolor=blue,
  urlcolor=blue
}

% Disable paragraph indentation and add space between paragraphs
\setlength{\parindent}{0pt}
\setlength{\parskip}{6pt}

% ----------------------------------------------------------------------
% Metadata helpers
% ----------------------------------------------------------------------
\newcommand{\code}[1]{\texttt{#1}}

\begin{document}

\begin{center}
  {\LARGE \bf Repository Layout}\\[0.5em]
  \today
\end{center}

\vspace{1em}

This PDF is generated from \code{writeup/tex/README.tex}. It documents the
folder hierarchy and how each directory feeds into the others.

\tableofcontents
\newpage

%%%%%%%%%%%%%%%%%%%%%%%%%%%%%%%%%%%%%%%%%%%%%%%%%%%%%%%%%%%%%%%%%%%%%%%%
\section{Directory Structure}
%%%%%%%%%%%%%%%%%%%%%%%%%%%%%%%%%%%%%%%%%%%%%%%%%%%%%%%%%%%%%%%%%%%%%%%%

\begin{description}[style=unboxed,leftmargin=0.7in]
  \item[\code{data/}] Raw inputs from external sources live in
        \code{data/raw}. Scripts under \code{src/} write processed panels
        to \code{data/processed}. Small samples kept under version control
        reside in \code{data/samples}.

  \item[\code{src/}] Stata build scripts that turn raw files into
        analysis-ready datasets. Every script sources \code{src/globals.do}
        for consistent paths and writes its output to
        \code{data/processed}.

  \item[\code{spec/}] Self-contained empirical specifications. Each
        \code{.do} file loads prepared panels from \code{data/processed},
        runs a model, and exports both tidy tables and raw diagnostics into
        \code{results/}.

  \item[\code{scripts/}] Small Python utilities that post-process Stata
        results. They merge standard errors, generate figures, and tidy
        output before it reaches the paper.

  \item[\code{results/}] Generated artefacts from all estimation scripts.
        The folder is split into \code{raw/}, \code{cleaned/}, and
        \code{figures/}. Clean tables are later copied into
        \code{writeup/}.

  \item[\code{writeup/}] Contains the \LaTeX{} source of the paper along
        with a Makefile. The build rules collect the cleaned tables from
        \code{results/cleaned} and compile the final PDF.
\end{description}

%%%%%%%%%%%%%%%%%%%%%%%%%%%%%%%%%%%%%%%%%%%%%%%%%%%%%%%%%%%%%%%%%%%%%%%%
\section{Interdependencies}
%%%%%%%%%%%%%%%%%%%%%%%%%%%%%%%%%%%%%%%%%%%%%%%%%%%%%%%%%%%%%%%%%%%%%%%%

The directories form a linear pipeline:

\begin{enumerate}[label=\arabic*)]
  \item \textbf{Raw data} enters under \code{data/raw}.
  \item Stata scripts in \code{src/} transform these files into structured
        panels stored in \code{data/processed}.
  \item Empirical models in \code{spec/} read the processed panels and
        write their outputs to \code{results/}.
  \item Python helpers in \code{scripts/} further clean these outputs so that
        the \code{writeup/} build can include them directly.
  \item The \LaTeX{} sources within \code{writeup/} assemble the final
        report using the cleaned tables and generated figures.
\end{enumerate}

Each stage depends only on the outputs of the previous one. Rebuilding a
panel or specification updates downstream folders without affecting
unrelated components.

\end{document}
