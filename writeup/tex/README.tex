% ======================================================================
% README.tex – Comprehensive repository documentation
% ----------------------------------------------------------------------
% This file is **source-of-truth** for the repository documentation.  It
% is compiled to a PDF (README.pdf) that serves as the canonical README
% at the project root.  Keep prose here and *only* a tiny stub in
% `README.md` so that GitHub still shows something reasonable when the
% repository is browsed on the web.
% ======================================================================

\documentclass[11pt]{article}

% ----------------------------------------------------------------------
% Packages
% ----------------------------------------------------------------------
\usepackage[margin=1in]{geometry}
\usepackage{hyperref}
\usepackage{xcolor}
\usepackage{enumitem}
\usepackage{minted}

\hypersetup{
  colorlinks=true,
  linkcolor=blue,
  urlcolor=blue,
  citecolor=blue
}

% Disable paragraph indentation and put vertical space between paragraphs
\setlength{\parindent}{0pt}
\setlength{\parskip}{6pt}

% ----------------------------------------------------------------------
% Metadata helpers
% ----------------------------------------------------------------------
\newcommand{\code}[1]{\texttt{#1}}

\begin{document}

\begin{center}
  {\LARGE \bf Repository Documentation}\\[0.5em]
  \today
\end{center}

\vspace{1em}

This PDF is generated from \code{writeup/tex/README.tex}.  To re-build it, run

\begin{minted}{bash}
pdflatex -interaction=nonstopmode writeup/tex/README.tex
\end{minted}

All content below explains \\the directory structure, build workflow, and
research pipeline for this project.

\tableofcontents

\newpage

%%%%%%%%%%%%%%%%%%%%%%%%%%%%%%%%%%%%%%%%%%%%%%%%%%%%%%%%%%%%%%%%%%%%%%%%
\section{High-level Overview}
%%%%%%%%%%%%%%%%%%%%%%%%%%%%%%%%%%%%%%%%%%%%%%%%%%%%%%%%%%%%%%%%%%%%%%%%

The repository is structured around three tasks that mirror the typical
empirical workflow of the project:

\begin{enumerate}[label=\arabic*)]
    \item \textbf{Data preparation} – Stata \code{.do} files in \code{src/}
          ingest the raw data and construct analysis-ready panels.
    \item \textbf{Estimation/specification} – individual empirical models
          live in \code{spec/}.  Each script draws on the prepared panels and
          writes tidy results.
    \item \textbf{Reporting} – Python and \LaTeX\ code inside \code{writeup/}
          turns the raw results into publication-quality tables, figures, and
          the final paper.
\end{enumerate}

A Makefile at \code{writeup/Makefile} orchestrates the reporting pipeline so
that a single command produces the paper PDF together with clean \LaTeX
tables.

%%%%%%%%%%%%%%%%%%%%%%%%%%%%%%%%%%%%%%%%%%%%%%%%%%%%%%%%%%%%%%%%%%%%%%%%
\section{Top-level Directory Layout}
%%%%%%%%%%%%%%%%%%%%%%%%%%%%%%%%%%%%%%%%%%%%%%%%%%%%%%%%%%%%%%%%%%%%%%%%

\begin{description}[style=unboxed,leftmargin=0.7in]
  \item[\code{data/}]     Raw inputs as shipped by external sources (\code{data/raw})
                          and processed panels generated by our build
                          scripts (\code{data/processed}).  Small samples that
                          *are* version-controlled reside in
                          \code{data/samples}.

  \item[\code{src/}]      Re-usable Stata build scripts.  Each file constructs a
                          particular panel (e.g. firm-level, worker-level)
                          starting entirely from the raw data.

  \item[\code{spec/}]     A "kitchen-sink" of empirical specifications.  Every
                          \code{.do} file is self-contained: it loads the
                          prepared data, runs the model(s), and writes two
                          types of artefacts:
                          \begin{enumerate}[label=\alph*)]
                            \item \code{results/cleaned/} – publishable
                                  \LaTeX\ tables.
                            \item \code{results/raw/}     – result dumps used
                                  for robustness checks and diagnostics.
                          \end{enumerate}

  \item[\code{py/}]       Lightweight Python helpers.  They post-process the
                          Stata output (e.g. merge standard errors, rename
                          variables) and generate additional figures that are
                          easier to code in Python.

  \item[\code{writeup/}]  Everything related to the paper: a
                          \code{Makefile}, intermediate build folder, final
                          PDF, and the \LaTeX\ sources.  All tables under
                          \code{results/cleaned} are copied here when the paper
                          builds.

  \item[\code{results/}]  Automatically generated outputs from the Stata and
                          Python code.  The directory is subdivided into
                          \code{raw/}, \code{cleaned/}, and \code{figures/}.
\end{description}

%%%%%%%%%%%%%%%%%%%%%%%%%%%%%%%%%%%%%%%%%%%%%%%%%%%%%%%%%%%%%%%%%%%%%%%%
\section{Workflow in Detail}
%%%%%%%%%%%%%%%%%%%%%%%%%%%%%%%%%%%%%%%%%%%%%%%%%%%%%%%%%%%%%%%%%%%%%%%%

\subsection*{1. Data preparation (Stata \code{src/})}

\begin{enumerate}
  \item All build scripts source \code{src/globals.do} first.  The file defines
        global macros (e.g. \code{\$raw\_data}, \code{\$processed\_data}) so
        that every subsequent script writes to a consistent location.
  \item Each build script checks whether its target panel already exists in
        \code{data/processed}.  If yes, nothing happens; if not, the script
        performs the necessary joins, merges, and reshapes.
\end{enumerate}

\subsection*{2. Estimation (Stata \code{spec/})}

Every specification script follows the same skeleton:

\begin{enumerate}
  \item Load the required panel(s) from \code{data/processed}.
  \item Run the main model plus any robustness checks.
  \item Dump full result matrices to \code{results/raw}.
  \item Write publication-ready tables to \code{results/cleaned} using
        a common \code{outreg2} template so that the look and feel are
        consistent across specifications.
\end{enumerate}

\subsection*{3. Reporting (\code{writeup/})}

\begin{enumerate}
  \item The Makefile builds all required tables by running the Python helpers
        in \code{writeup/py/}.  Each helper reads the raw CSV dumps and turns
        them into compact \LaTeX code under \code{results/cleaned}.
  \item After the tables are up-to-date, the Makefile compiles the main paper
        (\code{writeup/tex/results/consolidated-report.tex}) with \code{pdflatex}.
  \item A convenience target \code{make deploy} syncs the paper PDF and cleaned
        tables to an Overleaf-backed Dropbox folder so that collaborators can
        edit the manuscript online.
\end{enumerate}

%%%%%%%%%%%%%%%%%%%%%%%%%%%%%%%%%%%%%%%%%%%%%%%%%%%%%%%%%%%%%%%%%%%%%%%%
\section{Typical Usage}
%%%%%%%%%%%%%%%%%%%%%%%%%%%%%%%%%%%%%%%%%%%%%%%%%%%%%%%%%%%%%%%%%%%%%%%%

\subsection*{Building the full paper}

Run the following from the repository root (requires Stata, Python $\geq$ 3.9,
and \TeX\ Live):

\begin{minted}{bash}
# 1) Build data (only needs to run once)
stata -e src/build_firm_panel.do
stata -e src/build_user_panel.do

# 2) Run specifications
stata -e spec/firm_scaling.do
stata -e spec/worker_event_study.do

# 3) Build the paper
make -C writeup
\end{minted}

\subsection*{Quick rebuild after code changes}

If you only tweaked one specification, simply re-run that \code{.do} file
followed by \code{make -C writeup report}.  The existing panels and
unaffected tables will be left untouched, resulting in a much faster build.

%%%%%%%%%%%%%%%%%%%%%%%%%%%%%%%%%%%%%%%%%%%%%%%%%%%%%%%%%%%%%%%%%%%%%%%%
\section{Development Notes}
%%%%%%%%%%%%%%%%%%%%%%%%%%%%%%%%%%%%%%%%%%%%%%%%%%%%%%%%%%%%%%%%%%%%%%%%

\begin{itemize}
  \item \textbf{Version control.}  Large raw datasets are \emph{not}
        committed—only scripts and small samples live in Git. The canonical raw
        inputs reside on a shared network drive.

  \item \textbf{Reproducibility.}  All generated artefacts depend solely on
        committed code and the raw data referenced in \code{globals.do}.
        Running the three-step workflow above on a clean machine should
        reproduce every result.

  \item \textbf{Python dependencies.}  A minimal \code{requirements.txt} is
        provided at the repository root.  Create a fresh virtual environment
        and install with \code{pip install -r requirements.txt}.

  \item \textbf{Styling / linting.}  The project follows the standard Stata
        style guide and \code{black} for Python.
\end{itemize}

%%%%%%%%%%%%%%%%%%%%%%%%%%%%%%%%%%%%%%%%%%%%%%%%%%%%%%%%%%%%%%%%%%%%%%%%
\section{FAQ}
%%%%%%%%%%%%%%%%%%%%%%%%%%%%%%%%%%%%%%%%%%%%%%%%%%%%%%%%%%%%%%%%%%%%%%%%

\begin{description}[style=unboxed,leftmargin=0.6in]
  \item[\textsc{Q}:] "I changed a raw data file—why is the script not
        re-building?"\\
        \textbf{A}: Delete the corresponding processed file(s) in
        \code{data/processed} and rerun the build script.  The scripts only
        rebuild panels that do not yet exist.

  \item[\textsc{Q}:] "A table shows \code{(omitted)} for some coefficient."\\
        \textbf{A}: Stata omits perfectly collinear regressors.  Double-check
        the fixed-effect structure and ensure you are not including the same
        categorical variable twice.

  \item[\textsc{Q}:] "How do I add a new specification?"\\
        \textbf{A}:
        \begin{enumerate}
          \item Duplicate the template at \code{spec/template.do} (or any
                existing \code{.do} file).
          \item Point the script towards the relevant panel(s).
          \item Use the helper programs defined in \code{src/globals.do} to
                write cleaned \LaTeX tables.
        \end{enumerate}
\end{description}

\end{document}
