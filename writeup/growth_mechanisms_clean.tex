\documentclass[11pt]{article}
\usepackage{booktabs}
\usepackage{array}
\usepackage{multirow}
\usepackage{graphicx}
\usepackage[margin=1in]{geometry}
\usepackage{adjustbox}
\usepackage{threeparttable}
\usepackage{amssymb}
\usepackage{amsmath}

\title{Growth Mechanisms and Remote Work Productivity\\
\large Comparing Fixed Effects Specifications}
\author{Analysis Report}
\date{\today}

\begin{document}

\maketitle

\section{Introduction}

This analysis examines how firm growth interacts with remote work effects on productivity. We test whether the startup advantage in remote work productivity is explained by firm growth, distinguishing between:
\begin{itemize}
\item \textbf{Endogenous growth}: Raw firm growth that may be correlated with remote work success
\item \textbf{Exogenous growth}: Growth predicted by external factors (industry trends, location, market structure)
\end{itemize}

We present results using two fixed effects specifications to assess robustness.

\section{Methodology}

We estimate instrumental variables regressions of the form:
\begin{equation}
Y_{it} = \beta_1 \text{Remote}_i \times \text{Post}_t + \beta_2 \text{Remote}_i \times \text{Post}_t \times \text{Startup}_i + \gamma X_{it} + \alpha + \epsilon_{it}
\end{equation}

where:
\begin{itemize}
\item $Y_{it}$ is worker productivity (contributions per 100 days)
\item $X_{it}$ includes growth interactions
\item $\alpha$ represents fixed effects (specified below)
\end{itemize}

\textbf{Growth measures:}
\begin{enumerate}
\item Endogenous: Median split of raw post-COVID firm growth
\item Exogenous: Median split of growth predicted by industry/MSA trends, rent, and HHI
\end{enumerate}

\section{Results}

\subsection{Worker-Firm Fixed Effects}

This specification includes worker-firm pair fixed effects, controlling for time-invariant match quality between workers and firms.

% First table
\input{../results/cleaned/growth_mechanisms_worker_firm_fe.tex}

\subsection{Separate Fixed Effects}

This specification includes separate worker and firm fixed effects, allowing worker quality and firm characteristics to be controlled independently.

% Second table
\input{../results/cleaned/growth_mechanisms_separate_fe.tex}

\section{Key Findings}

\begin{enumerate}
\item \textbf{Baseline startup advantage}: 
   \begin{itemize}
   \item Worker-firm FE: 12.3 pp (p$<$0.05)
   \item Separate FE: 9.9 pp (p$<$0.10)
   \end{itemize}
   The stronger effect with worker-firm FE suggests match quality is important.

\item \textbf{Raw growth explains much of the advantage}: When controlling for endogenous (raw) growth, the startup effect becomes insignificant under both specifications (6.8 pp and 4.7 pp respectively).

\item \textbf{Advantage persists after controlling for external factors}: When controlling for exogenous growth, the startup advantage remains substantial, especially with worker-firm FE (11.9 pp, p$<$0.10).

\item \textbf{Interpretation}: The startup advantage in remote work productivity:
   \begin{itemize}
   \item Is correlated with overall firm growth performance
   \item Cannot be explained by external growth drivers alone
   \item Appears stronger when accounting for worker-firm matching
   \end{itemize}
\end{enumerate}

\section{Conclusion}

The results suggest that startups' remote work advantage reflects both their growth trajectory and their ability to implement remote work effectively. The persistence of the effect when controlling for externally-driven growth indicates that the advantage is not merely a mechanical result of being in favorable industries or locations. The importance of worker-firm fixed effects further suggests that successful remote work implementation involves matching with appropriate workers.

\end{document}