\documentclass[11pt]{article}
\usepackage[margin=1in]{geometry}
\usepackage{graphicx}
\usepackage{float}
\usepackage{amsmath}
\usepackage{booktabs}
\usepackage{caption}
\usepackage{subcaption}
\usepackage{listings}
\usepackage{xcolor}

% Define colors for code
\definecolor{codegreen}{rgb}{0,0.6,0}
\definecolor{codegray}{rgb}{0.5,0.5,0.5}
\definecolor{codepurple}{rgb}{0.58,0,0.82}
\definecolor{backcolour}{rgb}{0.95,0.95,0.92}

% Code formatting
\lstdefinestyle{mystyle}{
    backgroundcolor=\color{backcolour},   
    commentstyle=\color{codegreen},
    keywordstyle=\color{magenta},
    numberstyle=\tiny\color{codegray},
    stringstyle=\color{codepurple},
    basicstyle=\ttfamily\footnotesize,
    breakatwhitespace=false,         
    breaklines=true,                 
    captionpos=b,                    
    keepspaces=true,                 
    numbers=left,                    
    numbersep=5pt,                  
    showspaces=false,                
    showstringspaces=false,
    showtabs=false,                  
    tabsize=2
}
\lstset{style=mystyle}

\title{Productivity Spillovers from Role-Specific Hiring: \\
       Evidence from Local Projections}
\author{}
\date{}

\begin{document}

\maketitle

\section{Approach}

We estimate productivity spillovers from hiring employees in different roles using local projections (Jord\`{a} 2005). Local projections estimate impulse response functions (IRFs) by running separate regressions for each horizon, with the lead of the outcome variable as the dependent variable.

\subsection{Identification Strategy}

The identification strategy exploits the temporal structure of the data: current composition growth affects future productivity, eliminating reverse causality concerns. Since future productivity cannot cause current hiring decisions, the coefficient $\beta_h$ identifies the causal effect of composition shocks on productivity $h$ periods ahead.

\subsection{Specification}

For each horizon $h = 0, 1, 2, 3, 4$, we estimate:

\begin{equation}
\text{Productivity}_{i,f,t+h} = \sum_{j=1}^{7} \beta_{j,h} \cdot \text{RoleGrowth}_{j,f,t} + \alpha_{if} + \gamma_{t} + \varepsilon_{i,f,t+h}
\end{equation}

where:
\begin{itemize}
\item $i$ indexes individuals, $f$ indexes firms, $t$ indexes time periods (6-month intervals)
\item $h$ is the horizon (0-4 periods ahead, representing 0-2 years)
\item $\text{RoleGrowth}_{j,f,t}$ is the percentage growth in role $j$ hiring at firm $f$ in period $t$
\item $j$ indexes the 7 roles: Admin, Engineer, Finance, Marketing, Operations, Sales, Scientist
\item $\alpha_{if}$ are user$\times$firm fixed effects controlling for all time-invariant match-specific factors
\item $\gamma_t$ are time fixed effects controlling for aggregate productivity trends
\end{itemize}

The IRF for role $j$ is the sequence $\{\beta_{j,0}, \beta_{j,1}, \beta_{j,2}, \beta_{j,3}, \beta_{j,4}\}$.


\section{Data and Estimation}

The analysis uses pre-COVID panel data (2018-2022) with 177,353 user-firm-time observations. Composition growth variables measure percentage changes in role-specific hiring by firms. The outcome variable is individual productivity measured as percentile rank (1-100).

\subsection{Implementation}

Local projections are implemented by running separate regressions for each horizon. For horizon $h$, the dependent variable is productivity $h$ periods in the future. We create lead variables:

\begin{lstlisting}[language=bash, caption=Creating Lead Variables]
// Generate productivity leads for horizons 0-4
forvalues h = 0/4 {
    by user_id: gen F`h'_prod = total_contributions_q100[_n+`h']
}
\end{lstlisting}

Then estimate one regression per horizon:

\begin{lstlisting}[language=bash, caption=Local Projections Estimation]
forvalues h = 0/4 {
    reghdfe F`h'_prod pct_growth_Admin pct_growth_Engineer ///
        pct_growth_Finance pct_growth_Marketing ///
        pct_growth_Operations pct_growth_Sales pct_growth_Scientist, ///
        absorb(user_id#firm_id yh) vce(cluster user_id)
}
\end{lstlisting}

This approach estimates 5 separate regressions (one per horizon). The sequence of coefficients $\beta_{j,h}$ for each role $j$ forms the impulse response function.


\section{Results}

Table \ref{tab:results} shows coefficient estimates for all roles across horizons.

\begin{table}[H]
\centering
\caption{IRF Coefficient Estimates by Role and Horizon}
\label{tab:results}
\begin{tabular}{lrrrrr}
\toprule
Role & H0 & H1 & H2 & H3 & H4 \\
\midrule
Admin & -0.353 & -0.172 & -0.812*** & 0.754** & 0.113 \\
Engineer & -0.844** & 1.494*** & 1.419** & 0.783 & -0.051 \\
Finance & -0.591** & -0.557* & -0.015 & 0.460 & -0.254 \\
Marketing & 0.191 & 0.016 & 0.102 & 0.508 & -0.183 \\
Operations & 0.497** & 0.065 & -0.674** & -0.516 & 0.583 \\
Sales & 0.561** & 0.865** & 0.780** & 0.497 & 1.284** \\
Scientist & 0.067 & 0.655*** & 0.316 & -0.089 & 0.463 \\
\bottomrule
\end{tabular}
\begin{minipage}{\textwidth}
\footnotesize
Notes: * p$<$0.10, ** p$<$0.05, *** p$<$0.01. Coefficients show effect on productivity percentile from role-specific composition growth. All specifications include user$\times$firm and time fixed effects with standard errors clustered at user level.
\end{minipage}
\end{table}

\section{Individual Role IRFs}

Figures \ref{fig:admin}-\ref{fig:scientist} show the individual IRF plots for each role with 95\% confidence intervals.

\begin{figure}[H]
\centering
\includegraphics[width=0.8\textwidth]{../results/composition_irfs_all7/clean_irf_Admin.png}
\caption{Admin Hiring IRF}
\label{fig:admin}
\end{figure}

\begin{figure}[H]
\centering
\includegraphics[width=0.8\textwidth]{../results/composition_irfs_all7/clean_irf_Engineer.png}
\caption{Engineer Hiring IRF}
\label{fig:engineer}
\end{figure}

\begin{figure}[H]
\centering
\includegraphics[width=0.8\textwidth]{../results/composition_irfs_all7/clean_irf_Finance.png}
\caption{Finance Hiring IRF}
\label{fig:finance}
\end{figure}

\begin{figure}[H]
\centering
\includegraphics[width=0.8\textwidth]{../results/composition_irfs_all7/clean_irf_Marketing.png}
\caption{Marketing Hiring IRF}
\label{fig:marketing}
\end{figure}

\begin{figure}[H]
\centering
\includegraphics[width=0.8\textwidth]{../results/composition_irfs_all7/clean_irf_Operations.png}
\caption{Operations Hiring IRF}
\label{fig:operations}
\end{figure}

\begin{figure}[H]
\centering
\includegraphics[width=0.8\textwidth]{../results/composition_irfs_all7/clean_irf_Sales.png}
\caption{Sales Hiring IRF}
\label{fig:sales}
\end{figure}

\begin{figure}[H]
\centering
\includegraphics[width=0.8\textwidth]{../results/composition_irfs_all7/clean_irf_Scientist.png}
\caption{Scientist Hiring IRF}
\label{fig:scientist}
\end{figure}

\section{Combined Results}

Figure \ref{fig:combined} shows all seven roles on a single plot for comparison.

\begin{figure}[H]
\centering
\includegraphics[width=\textwidth]{../results/composition_irfs_all7/clean_combined_all7.png}
\caption{Combined IRFs for All Roles}
\label{fig:combined}
\end{figure}


\end{document}