\documentclass[11pt]{article}
\usepackage{booktabs}
\usepackage{array}
\usepackage{multirow}
\usepackage{graphicx}
\usepackage[margin=1in]{geometry}
\usepackage{adjustbox}
\usepackage{threeparttable}
\usepackage{amssymb}
\usepackage{amsmath}

\title{Growth Mechanisms Horse Race Analysis\\
\large Remote Work Productivity Effects}
\author{Analysis Report}
\date{\today}

\begin{document}

\maketitle

\section{Introduction}

This document presents a systematic horse race analysis testing different mechanisms that may drive the heterogeneous effects of remote work on productivity. We examine how firm growth, rent levels, and market concentration (HHI) interact with remote work adoption to affect worker productivity.

\section{Methodology}

\subsection{Empirical Specification}

We estimate the following general specification:

\begin{equation}
Y_{it} = \beta_1 \text{Remote}_i \times \text{Post}_t + \beta_2 \text{Remote}_i \times \text{Post}_t \times \text{Startup}_i + \gamma X_{it} + \alpha_{i,f} + \delta_t + \epsilon_{it}
\end{equation}

where:
\begin{itemize}
\item $Y_{it}$ is worker productivity (contributions per 100 days)
\item $\text{Remote}_i$ indicates remote work eligibility
\item $\text{Post}_t$ indicates the post-COVID period
\item $\text{Startup}_i$ indicates whether the firm is a startup
\item $X_{it}$ includes interaction terms with firm characteristics
\item $\alpha_{i,f}$ are worker-firm fixed effects
\item $\delta_t$ are time fixed effects
\end{itemize}

\subsection{Mechanisms Tested}

We test two types of growth interactions:
\begin{enumerate}
\item \textbf{Endogenous Growth}: Interactions with raw post-COVID firm growth (above/below median)
\item \textbf{Exogenous Growth}: Interactions with growth residualized on rent and HHI to isolate idiosyncratic firm performance
\end{enumerate}

\section{Results}

% Include the generated table
\input{../results/cleaned/growth_mechanisms_focused.tex}

\section{Key Findings}

\subsection{Main Results}

\begin{enumerate}
\item \textbf{Baseline Effect}: In the baseline specification (Column 1), remote work has a negative effect on productivity for non-startups (-9.26 pp) but a positive net effect for startups (+3.19 pp = -9.26 + 12.45).

\item \textbf{Endogenous Growth} (Column 2): When interacting with raw firm growth, the startup effect becomes smaller and statistically insignificant (5.56 pp, p>0.10). This suggests that high-growth firms may drive some of the baseline remote work benefits.

\item \textbf{Exogenous Growth} (Column 3): When using growth residualized on rent and HHI, the startup effect remains strong and significant (13.89 pp). This indicates that idiosyncratic firm performance (beyond market and location factors) does not explain the remote work benefits for startups.
\end{enumerate}

\subsection{Interpretation}

The contrasting results between endogenous and exogenous growth specifications reveal important insights:
\begin{itemize}
\item Raw firm growth partially captures the remote work effect, suggesting some correlation between firms that benefit from remote work and those experiencing high growth
\item However, when we isolate idiosyncratic firm performance (exogenous growth), the startup advantage in remote work productivity persists
\item This indicates that the remote work benefits for startups are not simply a byproduct of their growth trajectory or market conditions
\item The results support the hypothesis that startups have inherent characteristics (e.g., flexible work culture, digital-native processes) that enable them to leverage remote work more effectively
\end{itemize}

\section{Technical Notes}

\subsection{Identification}
We use instrumental variables (IV) estimation with pre-determined remote work policies as instruments. The first-stage F-statistics (>100) indicate strong instruments across all specifications.

\subsection{Sample}
The analysis uses the pre-COVID panel of workers and firms, focusing on within-firm-worker variation in productivity around the COVID shock.

\subsection{Growth Residualization}
The exogenous growth measure is created by regressing raw firm growth on leave-one-out industry and MSA growth rates, along with rent and HHI tiles, with an R-squared of 0.14. This removes industry trends, location effects, and market structure influences while preserving idiosyncratic firm performance.

\end{document}